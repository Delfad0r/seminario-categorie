\section*{Monadi}

\begin{frame}[fragile]{\secname}{Esempio motivazionale}
Una funzione di tipo \h/a -> Maybe b/ è concettualmente una funzione da \h/a/ a \h/b/ che potrebbe fallire.
\pause
\begin{itemize}[<+->]
\item \h/head :: List a -> Maybe a/\\
Restituisce il primo elemento della lista, fallisce se la lista è vuota.
\item \h/sqrt :: Double -> Maybe Double/\\
Restituisce la radice quadrata, fallisce se l'argomento è negativo.
\item \h/reciprocal :: Double -> Maybe Double/\\
Restituisce il reciproco, fallisce se l'argomento è nullo.
\end{itemize}
\end{frame}

\begin{frame}[fragile]{\secname}{Esempio motivazionale}
Vogliamo scrivere una funzione

\begin{haskellcode}
solveQuartic
    :: Double -> Double -> Double -> Maybe Double
\end{haskellcode}

che prende in input tre numeri reali \h/a/, \h/b/, \h/c/ e restituisce una soluzione reale dell'equazione $\h/a/x^4+\h/b/x^2+\h/c/=0$, in particolare
\[
\sqrt{\frac{-\h/b/+\sqrt{\h/b/^2-4\h/a/\h/c/}}{2\h/a/}}.
\]
\pause
Se $\h/a/=0$ oppure uno dei radicandi è negativo, \h/solveQuartic/ restituirà \h/Nothing/.
\end{frame}

\begin{frame}[fragile]{\secname}{Esempio motivazionale}
\begin{block}{Primo tentativo}
\begin{haskellcode}
solveQuartic
    :: Double -> Double -> Double -> Maybe Double
solveQuartic a b c =
    case sqrt (b ^ 2 - 4 * a * c) of
        Nothing -> Nothing
        Just d -> case reciprocal (2 * a) of
            Nothing -> Nothing
            Just a' -> sqrt (a' * (-b + d))
\end{haskellcode}
\pause

Il codice sembra ridondante (per ogni funzione, deve assicurarsi che non sia fallita).
\end{block}
\end{frame}

\begin{frame}[fragile]{\secname}{Esempio motivazionale}
\begin{block}{Problema}
Due funzioni \h/g :: a -> Maybe b/ e \h/h :: b -> Maybe c/ non si possono comporre.
\end{block}
\pause
\begin{block}{Soluzione?}
Comporre \h/g :: a -> Maybe b/ con \h/map h :: Maybe b -> Maybe (Maybe c)/
\end{block}
\pause
\begin{block}{Problema!}
Problema: il risultato di \h/map h . g/ è di tipo \h/Maybe (Maybe c)/, mentre noi vorremmo \h/Maybe c/.
\end{block}
\pause
\begin{block}{Soluzione!}
Monadi!
\end{block}
\end{frame}

\begin{frame}[fragile]{\secname}{Definizione}
Una monade in Haskell è una monade
\[
(\h/m/,\h/unit/,\h/join/):\Hask{}\to\Hask{}.
\]
\vspace{-0.5cm}
\begin{itemize}[<+(1)->]
\item \h/m/ è un funtore, dotato di \h/map :: (a -> b) -> m a -> m b/.
\item \h/unit/ è una funzione \h/:: a -> m a/.
\item \h/join/ è una funzione \h/:: m (m a) -> m a/.
\end{itemize}
\pause

\begin{haskellcode}
class Functor m => Monad (m :: * -> *) where
    unit :: a -> m a
    join :: m (m a) -> m a
\end{haskellcode}
\end{frame}

\begin{frame}[fragile]{\secname}{Definizione}
\begin{overlayarea}{\textwidth}{0.7\textheight}
Per essere una monade, \h/m/ deve soddisfare (oltre alle proprietà funtoriali):
\begin{onlyenv}<+(1)>
\begin{itemize}
\item per ogni tipo \h/a/,
\begin{haskellcode}
join . join == join . map join
    :: m (m (m a)) -> m a
\end{haskellcode}
\begin{diagram}
\h/m (m (m a))/\rar{\h/map join/}\dar["\h/join/" above]&\h/m (m a)/\dar["\h/join/" above]\\
\h/m (m a)/\rar{\h/join/}& \h/m a/
\end{diagram}
\end{itemize}
\end{onlyenv}
\begin{onlyenv}<+(1)>
\begin{itemize}
\item per ogni tipo \h/a/,
\begin{haskellcode}
join . unit == join . map unit == id
    :: m a -> m a
\end{haskellcode}
\begin{diagram}
\h/m a/\rar{\h/unit/}\ar[dr,"\h/id/"]&\h/m (m a)/\dar["\h/join/" above]&\h/m a/\lar["\h/map unit/" above]\ar[dl,"\h/id/" above]\\
&\h/m a/
\end{diagram}
\end{itemize}
\end{onlyenv}
\end{overlayarea}
\end{frame}

\begin{frame}[fragile]{\secname}{La monade \h/Maybe/}
Il funtore \h/Maybe/ ha una struttura di monade.
\pause

\begin{haskellcode}
instance Monad Maybe where
    unit x = Just x
    join Nothing = Nothing
    join (Just x) = x
\end{haskellcode}
\pause 
\begin{runhaskell}
>>  join Nothing
    Nothing
>>  join (Just Nothing)
    Nothing
>>  join (Just (Just 3))
    Just 3
\end{runhaskell}
\end{frame}

\begin{frame}[fragile]{\secname}{L'operatore \h/(>==)/}
\begin{haskellcode}
(>>=) :: Monad m => m a -> (a -> m b) -> m b
x >>= g = join (map g x)
\end{haskellcode}

\pause
L'operatore \h/(>>=)/ consente di comporre morfismi nella categoria di Kleisli di \h/m/ (i.e. funzioni \h/:: a -> m b/).

\pause
Consideriamo due funzioni \h/g :: a -> m b/, \h/h :: b -> m c/.
\pause
\begin{diagram}[column sep = small]
&\h/m (m b)/\ar[dr,"\h/join/" above]&&\h/m (m c)/\ar[dr,"\h/join/" above]\\
\h/m a/\ar[rr,"\h/\x -> x >>= g/"]\ar[ru,"\h/map g/" above]\ar[rrrr,"\h/\x -> x >>= g >>= h/" below,out=-15,in=-165]&&\h/m b/\ar[rr,"\h/\x -> x >>= h/"]\ar[ru, "\h/map h/" above]&&\h/m c/
\end{diagram}
\end{frame}

\begin{frame}[fragile]{\secname}{L'operatore \h/(>==)/}
Nel caso di \h/Maybe/, l'operatore \h/(>>=)/ si comporta come segue.

\pause
\begin{haskellcode}
(>>=) :: Maybe a -> (a -> Maybe b) -> Maybe b
Nothing >>= g = Nothing
(Just x) >>= g = g x
\end{haskellcode}

\pause
\begin{minipage}[t]{0.58\textwidth}
\begin{runhaskell}
>>  let g x = head x >>= head
>>  head []
    Nothing
>>  head [[], [1]]
    Just []
>>  head [[1], []]
    Just [1]
\end{runhaskell}
\end{minipage}
\hfill\begin{minipage}[t]{0.38\textwidth}
\begin{runhaskell}
>>
>>  g []
    Nothing
>>  g [[], [1]]
    Nothing
>>  g [[1], []
    Just 1
\end{runhaskell}
\end{minipage}
\end{frame}

\begin{frame}[fragile]{\secname}{Esempio motivazionale, parte 2}
Possiamo ora scrivere la funzione \h/solveQuartic/ in modo più conciso.
\pause

\begin{haskellcode}
solveQuartic
    :: Double -> Double -> Double -> Maybe Double
solveQuartic a b c =
    sqrt (b ^ 2 - 4 * a * c) >>= \d  ->
    reciprocal (2 * a)       >>= \a' ->
    sqrt (a' * (-b + d))
\end{haskellcode}
\end{frame}

\begin{frame}[fragile]{\secname}{Esempio motivazionale, parte 2}
\begin{block}{\emph{Do notation}}
\begin{minipage}[t]{0.48\textwidth}
\vspace{-0.3cm}
\begin{haskellcode}
    g x   >>= \y  ->
    h x y >>= \z  ->
    g' z  >>= \y' ->
    unit (y + y')
\end{haskellcode}
\end{minipage}
\hfill\pause
\begin{minipage}[t]{0.48\textwidth}
\vspace{-0.3cm}
\begin{haskellcode}
do  y  <- g x
    z  <- h x y
    y' <- g' z
    unit (y + y')
\end{haskellcode}
\end{minipage}
\end{block}
\pause

Con la \emph{do notation}, la funzione \h/solveQuartic/ assume la sua forma definitiva.
\pause

\begin{haskellcode}
solveQuartic
    :: Double -> Double -> Double -> Maybe Double
solveQuartic a b c = do
    d  <- sqrt (b ^ 2 - 4 * a * c)
    a' <- reciprocal (2 * a)
    sqrt (a' * (-b + d))
\end{haskellcode}
\end{frame}

\begin{frame}[fragile]{\secname}{La monade \h/List/}
Il funtore \h/List/ ha una struttura di monade.
\pause

\begin{haskellcode}
instance Monad List where
    unit x = [x]
    join Nil = Nil
    join (Cons l ls) = l ++ (join ls)
\end{haskellcode}
\pause

\begin{runhaskell}
>>  unit True
    [True]
>>  join [[1, 2], [], [2, 3]]
    [1, 2, 2, 3]
>>  [1, 2] >>= \x -> [x, -x]
    [1, -1, 2, -2]
\end{runhaskell}
\end{frame}

\begin{frame}[fragile]{\secname}{Versatilità}

Una funzione di tipo \h/a -> List b/ è concettualmente una \emph{multi-valued function} da \h/a/ a \h/b/.
\pause

\begin{haskellcode}
sqrt' :: Double -> List Double
sqrt' x = case sqrt x of
    Nothing -> []
    Just y  -> [y, -y]

reciprocal' :: Double -> List Double
reciprocal' x = case reciprocal x of
    Nothing -> []
    Just y  -> [y]
\end{haskellcode}
\end{frame}

\begin{frame}[fragile]{\secname}{Versatilità}
\begin{overlayarea}{\textwidth}{0.7\textheight}
Supponiamo ora di voler modificare \h/solveQuartic/ in modo che restituisca tutte le soluzioni (reali).

\begin{onlyenv}<2-3>
\begin{block}{Codice vecchio}
\begin{haskellcode}
solveQuartic
    :: Double -> Double -> Double -> Maybe Double
solveQuartic  a b c = do
    d  <- sqrt  (b ^ 2 - 4 * a * c)
    a' <- reciprocal  (2 * a)
    sqrt  (a' * (-b + d))
\end{haskellcode}

\begin{onlyenv}<3>
\begin{runhaskell}
>>  solveQuartic 1 (-1) 1
    Nothing
>>  solveQuartic 1 (-3) 1
    Just 1.618
\end{runhaskell}
\end{onlyenv}
\end{block}
\end{onlyenv}

\begin{onlyenv}<4-5>
\begin{block}{Codice nuovo}
\begin{haskellcode}
solveQuartic'
    :: Double -> Double -> Double -> List Double
solveQuartic' a b c = do
    d  <- sqrt' (b ^ 2 - 4 * a * c)
    a' <- reciprocal' (2 * a)
    sqrt' (a' * (-b + d))
\end{haskellcode}

\begin{onlyenv}<5>
\begin{runhaskell}
>>  solveQuartic' 1 (-1) 1
    []
>>  solveQuartic' 1 (-3) 1
    [1.618, -1.618, 0.618, -0.618]
\end{runhaskell}
\end{onlyenv}
\end{block}
\end{onlyenv}
\end{overlayarea}
\end{frame}