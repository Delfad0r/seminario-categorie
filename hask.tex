\begin{frame}[fragile]
Un programma in Haskell opera dentro la categoria \Hask{}.
\begin{itemize}[<+(1)->]
\item Gli oggetti di questa categoria sono i tipi.
\item Dati due tipi \h/a/, \h/b/, i morfismi $\Hom(\h/a/, \h/b/)$ sono le funzioni \h/:: a -> b/.
\item Per ogni tipo \h/a/, l'identità è la funzione \h/id = (\x -> x) :: a -> a/.
\item La composizione di morfismi è esattamente la composizione di funzioni.

\begin{haskellcode}
(.) :: (b -> c) -> (a -> b) -> a -> c
h . g = \x -> h (g x)
\end{haskellcode}

\end{itemize}
\end{frame} 

\begin{frame}[fragile]
\begin{overlayarea}{\textwidth}{0.6\textheight}
Come creare nuovi tipi?

\begin{itemize}
\item<2-> \haskell/data Bool = False | True/

\begin{onlyenv}<3>
\vspace{0.2cm}
\h/Bool/ è un \emph{tipo}, mentre \h/False/ e \h/True/ sono \emph{valori}:

\begin{haskellcode}
False :: Bool
True  :: Bool
\end{haskellcode}
\end{onlyenv}

\item<4-> \haskell/data Coordinate = Long Double | Lat Double/
\begin{onlyenv}<5>
\h/Long/ e \h/Lat/ sono \emph{data constructors}:

\begin{haskellcode}
Long    :: Double -> Coordinate
Lat 3.0 :: Coordinate
\end{haskellcode}
\end{onlyenv}
\item<6-> \haskell/data Maybe a = Nothing | Just a/
\begin{onlyenv}<7>
\h/Maybe/ è un \emph{type constructor}: prende un tipo \h/a/ e restituisce il tipo \h/Maybe a/. Diciamo che ha \emph{kind} \h/Maybe :: * -> */, laddove i tipi semplici hanno \emph{kind} \h/a :: */.

\begin{haskellcode}
Nothing :: Maybe a
Just    :: a -> Maybe a
Just 3  :: Maybe Int
\end{haskellcode}
\end{onlyenv}
\item<8-> \haskell/data List a = Nil | Cons a (List a)/
\begin{onlyenv}<9>
Di nuovo, \h/List :: * -> */ è un costruttore di tipi, mentre ad esempio \h/List Bool :: */ è un tipo.

\begin{haskellcode}
Nil :: List a
Cons :: a -> List a -> List a
Cons 3 Nil :: List Int
Cons 3 (Cons 1 Nil) :: List Int
\end{haskellcode}
\end{onlyenv}
\begin{onlyenv}<10>
Utilizzeremo la notazione abbreviata \h/[3, 1]/ in luogo di \h/Cons 3 (Cons 1 Nil)/.
\end{onlyenv}
\end{itemize}
\end{overlayarea}

\end{frame}

\begin{frame}[fragile]
\begin{itemize}
\item È facile vedere che \Hask{} ammette prodotti e coprodotti (finiti).

\begin{haskellcode}
data Pair a b = Pair a b
data Either a b = Left a | Right b
\end{haskellcode}

\item \Hask{} ammette un oggetto terminale.

\begin{haskellcode*}{linenos=false}
data Unit = Unit
\end{haskellcode*}

\item Esiste anche un oggetto iniziale (\h/Void/), la cui definizione richiede un po' di tecnicismi. Si tratta di fatto del tipo vuoto.
\end{itemize}
\end{frame}