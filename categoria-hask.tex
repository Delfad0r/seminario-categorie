\section*{La categoria \protect\Hask{}}

\begin{frame}[fragile]{\secname}{Oggetti e morfismi}
Un programma in Haskell opera dentro la categoria \Hask{}.
\begin{itemize}[<+(1)->]
\item Gli oggetti di questa categoria sono i tipi.
\item Dati due tipi \h/a/, \h/b/, i morfismi $\Hom(\h/a/, \h/b/)$ sono le funzioni \h/:: a -> b/.
\item Per ogni tipo \h/a/, l'identità è la funzione \h/id = (\x -> x) :: a -> a/.
\item La composizione di morfismi è esattamente la composizione di funzioni.

\begin{haskellcode}
(.) :: (b -> c) -> (a -> b) -> (a -> c)
h . g = \x -> h (g x)
\end{haskellcode}
\end{itemize}
\end{frame}

\begin{frame}[fragile]{\secname}{Proprietà categoriali}
\begin{itemize}[<+->]
\item \Hask{} ammette prodotti e coprodotti (finiti).

\begin{haskellcode}
data Pair a b = Pair a b
data Either a b = Left a | Right b
\end{haskellcode}

\item \Hask{} ammette un oggetto terminale.

\begin{haskellcode}
data Unit = Unit
\end{haskellcode}

\item Esiste anche un oggetto iniziale (\h/Void/), la cui definizione richiede un po' di tecnicismi. Si tratta di fatto del tipo vuoto.
\end{itemize}
\end{frame}

\section*{Funtori}

\begin{frame}[fragile]{\secname}{Definizione}
Un funtore in Haskell è un endofuntore $\h/f/:\Hask{}\to\Hask{}$.

\begin{itemize}[<+(1)->]
\item A ogni tipo \h/a/ associa un tipo \h/f a/.
\item A ogni funzione di tipo \h/a -> b/ associa una funzione di tipo \h/f a -> f b/.
\item Deve preservare \emph{identità} e \emph{composizione}.
\end{itemize}
\end{frame}

\begin{frame}[fragile]{\secname}{Definizione}

\begin{overlayarea}{\textwidth}{0.7\textheight}
\begin{haskellcode}
class Functor (f :: * -> *) where
    map :: (a -> b) -> (f a -> f b)
\end{haskellcode}

\pause
Per essere un funtore, \h/f/ deve soddisfare:
\begin{onlyenv}<+(1)>
\begin{itemize}
\item per ogni tipo \h/a/,
\haskell/map (id :: a -> a) == (id :: f a -> f a)/
\end{itemize}
\end{onlyenv}
\begin{itemize}[<+(1)>]
\item
per ogni \h/g :: a -> b/, \h/h :: b -> c/,
\haskell/map (h . g) == map h . map g/
\begin{diagram}
\h/f a/\rar{\h/map g/}\ar[dr,"\h/map (h . g)/" below]&\h/f b/\ar[d,"\h/map h/" above]\\
&\h/f c/
\end{diagram}
\end{itemize}
\end{overlayarea}
\end{frame}

\begin{frame}[fragile]{\secname}{Il funtore \h/Maybe/}
\begin{overlayarea}{\textwidth}{0.7\textheight}
\begin{haskellcode}
data Maybe a = Nothing | Just a

instance Functor Maybe where
    map g Nothing = Nothing
    map g (Just x) = Just (g x)
\end{haskellcode}

\begin{onlyenv}<2-4>
Un oggetto di tipo \h/Maybe a/ è una scatola che può:
\pause
\begin{itemize}[<+(1)->]
\item essere vuota, nel qual caso si indica con \h/Nothing/;
\item contenere un oggetto \h/x :: a/, nel qual caso si indica con \h/Just x/.
\end{itemize}
\end{onlyenv}

\begin{onlyenv}<5-7>
\h/Maybe/ è un funtore.
\pause
\begin{itemize}
\item<+(1)-> A ogni tipo \h/a/ associa il tipo \h/Maybe a/.
\item<+(1)-> A ogni funzione \h/g :: a -> b/ associa la funzione \h/map g :: Maybe a -> Maybe b/, la quale:
\begin{itemize}
\item manda \h/Nothing/ in \h/Nothing/;
\item manda \h/Just x/ in \h/Just (g x)/.
\end{itemize}
\end{itemize}
\end{onlyenv}

\begin{onlyenv}<8>
\begin{runhaskell}
>>  map (\x -> x + 1) Nothing
    Nothing
>>  map (\x -> x + 1) (Just 3)
    Just 4
>>  map (\x -> True) (Just 3)
    Just True
\end{runhaskell}
\end{onlyenv}
\end{overlayarea}
\end{frame}

\begin{frame}[fragile]{\secname}{Il funtore \h/List/}
\begin{overlayarea}{\textwidth}{0.7\textheight}
\begin{haskellcode}
data List a = Nil | Cons a (List a)

instance Functor List where
    map g Nil = Nil
    map g (Cons x l) = Cons (g x) (map g l)
\end{haskellcode}

\begin{onlyenv}<2-5>
Il tipo \h/List a/ è definito ricorsivamente. Una lista può:
\pause
\begin{itemize}[<+(1)->]
\item essere vuota, nel qual caso si indica con \h/Nil/;
\item iniziare con un oggetto \h/x :: a/ seguito dal resto della lista \h/l :: List a/, nel qual caso si indica con \h/Cons x l/.
\end{itemize}
\pause
Useremo la notazione abbreviata \h/[1, 2, 3]/ in luogo di
\h/Cons 1 (Cons 2 (Cons 3 Nil))/.
\end{onlyenv}

\begin{onlyenv}<6-8>
\h/List/ è un funtore.
\pause
\begin{itemize}[<+(1)->]
\item A ogni tipo \h/a/ associa il tipo \h/List a/.
\item A ogni funzione \h/g :: a -> b/ associa la funzione \h/map g :: List a -> List b/, che applica \h/g/ a ogni elemento della lista in input.
\end{itemize}
\end{onlyenv}

\begin{onlyenv}<9>
\begin{runhaskell}
>>  map (\x -> x + 1) []
    []
>>  map (\x -> x + 1) [1, 2, 3]
    [2, 3, 4]
>>  map (\x -> True) [1, 2, 3]
    [True, True, True]
\end{runhaskell}
\end{onlyenv}
\end{overlayarea}
\end{frame}

\begin{frame}[fragile]{\secname}{Il funtore \h/(->) a/}
\begin{haskellcode}
instance Functor ((->) a) where
    map g h = g . h
\end{haskellcode}
\pause

Per ogni tipo \h/a/, \h/(->) a/ è un funtore (concettualmente, si tratta del funtore $\Hom(\h/a/,-)$).
\begin{itemize}[<+(1)->]
\item A ogni tipo \h/b/ associa il tipo \h/a -> b/.
\item A ogni funzione \h/g :: b -> c/ associa la funzione \h/map g :: (a -> b) -> (a -> c)/, che è semplicemente la composizione con \h/g/.
\end{itemize}
\end{frame}

\section*{Trasformazioni naturali}

\begin{frame}[fragile]{\secname}{Funzioni polimorfe}
\begin{itemize}[<+->]
\item Dati due funtori \h/f/, \h/f'/, una trasformazione naturale da \h/f/ a \h/f'/ è il dato di una funzione \h/g :: f a -> f' a/ per ogni tipo \h/a/.
\item In altre parole, è una funzione polimorfa.
\item Teoremi profondi garantiscono che ogni funzione polimorfa soddisfa automaticamente le condizioni di naturalità.
\begin{diagram}
\h/f a/\rar{\h/g/}\dar["\h/map h/" above]&\h/f' a/\dar["\h/map h/" above]\\
\h/f b/\rar{\h/g/}&\h/f' b/
\end{diagram}
\end{itemize}
\end{frame}

\begin{frame}[fragile]{\secname}{Funzioni polimorfe}
\begin{overlayarea}{\textwidth}{0.7\textheight}
\begin{haskellcode}
head :: List a -> Maybe a
head Nil = Nothing
head (Cons x xs) = Just x
\end{haskellcode}

\begin{onlyenv}<2>
Possiamo verificare che \h/head/ è naturale, cioè
\haskell/map h . head = head . map h/
per ogni \h/h :: a -> b/.
\end{onlyenv}

\begin{onlyenv}<3>
\begin{runhaskell}
>>  let h = \x -> x + 1
>>  (map h . head) []
    Nothing
>>  (head . map h) []
    Nothing
>>  (map h . head) [1, 3, 5]
    Just 2
>>  (head . map h) [1, 3, 5]
    Just 2
\end{runhaskell}
\end{onlyenv}
\end{overlayarea}
\end{frame}

\begin{frame}[fragile]{\secname}{Lemma di Yoneda}
\begin{block}{Lemma di Yoneda in \Hask{}}
Sia $\h/a/\in\Hask{}$ un tipo, $\h/f/:\Hask{}\to\Hask{}$ un funtore. Allora esiste una corrispondenza biunivoca fra \h/f a/ e le trasformazioni naturali $\h/(->) a/\Rightarrow\h/f/$:
\[
\h/(a -> b) -> f b/\simeq\h/f a/
\]
\end{block}
\pause
\begin{block}{Corollario}
Ogni funzione \h/:: (a -> b) -> f a/ è necessariamente della forma \h/\g -> map g x/ per un qualche \h/x :: f a/.
\end{block}
\end{frame}