\section*{Monadi}

\begin{frame}[fragile]{\secname}{Funzioni con contesto}
Molto spesso, funzioni di tipo \h/a -> f b/ possono essere interpretate come ``funzioni'' \h/:: a -> b/ con qualche proprietà.
\pause
\begin{itemize}[<+->]
\item Funzioni \h/:: a -> Maybe b/ sono concettualmente funzioni che possono fallire. Ad esempio,
\haskell/head :: List a -> Maybe a/
restituisce il primo elemento della lista, ma fallisce se la lista è vuota.
\item Funzioni \h/:: a -> List b/ sono concettualmente \emph{multi-valued functions}. Ad esempio,
\haskell/sqrt :: Double -> List Double/
restituisce 0, 1 o 2 valori a seconda del segno dell'input.
\end{itemize}
\end{frame}

\begin{frame}[fragile]{\secname}{Esempio motivazionale}
Supponiamo di avere due funzioni

\begin{haskellcode}
father :: String -> Maybe String
mother :: String -> Maybe String
\end{haskellcode}
\pause

\begin{runhaskell}
>>  mother "Sansa"
    Just "Catelyn"
>>  father "Sansa"
    Just "Eddard"
>>  mother "Eddard"
    Nothing
\end{runhaskell}
\end{frame}

\begin{frame}[fragile]{\secname}{Esempio motivazionale}
Vogliamo scrivere una funzione

\begin{haskellcode}
maternalGF :: String -> Maybe String
\end{haskellcode}

che restituisce il nonno materno.
\pause
\begin{block}{Implementazione}
\begin{haskellcode}
maternalGF x = case mother x of
    Nothing -> Nothing
    Just mx -> father mx
\end{haskellcode}
\end{block}
\pause

\begin{runhaskell}
>>  maternalGF "Sansa"
    Just "Hoster"
>>  maternalGF "Eddard"
    Nothing
\end{runhaskell}
\end{frame}

\begin{frame}[fragile]{\secname}{Esempio motivazionale}
\begin{block}{Problema}
Questo metodo diventa impraticabile al crescere della complessità.
\pause

\begin{haskellcode}
maternalGGGF :: String -> Maybe String
maternalGGGF x = case mother x of
    Nothing -> Nothing
    Just mx -> case father mx of
        Nothing  -> Nothing
        Just fmx -> case father fmx of
            Nothing   -> Nothing
            Just ffmx -> father ffmx
\end{haskellcode}
\end{block}
\end{frame}

\begin{frame}[fragile]{\secname}{Esempio motivazionale}
\begin{block}{Problema}
Il problema nasce dall'impossibilità di comporre due funzioni di tipo \h/String -> Maybe String/.
\end{block}
\pause
\begin{block}{Soluzione?}
Comporre \h/map father :: Maybe String -> Maybe (Maybe String)/ con \h/mother :: String -> Maybe String/.
\end{block}
\pause
\begin{block}{Problema!}
La composizione \h/map father . mother/ restituisce un \h/Maybe (Maybe String)/, invece che \h/Maybe String/.
\end{block}
\pause
\begin{block}{Soluzione!}
Monadi!
\end{block}
\end{frame}

\begin{frame}[fragile]{\secname}{Definizione}
Una monade in Haskell è una monade
\[
(\h/m/,\h/unit/,\h/join/):\Hask{}\to\Hask{}.
\]
\vspace{-0.5cm}
\begin{itemize}[<+(1)->]
\item \h/m/ è un funtore, dotato di \h/map :: (a -> b) -> m a -> m b/.
\item \h/unit/ è una funzione \h/:: a -> m a/.
\item \h/join/ è una funzione \h/:: m (m a) -> m a/.
\end{itemize}
\pause

\begin{haskellcode}
class Functor m => Monad (m :: * -> *) where
    unit :: a -> m a
    join :: m (m a) -> m a
\end{haskellcode}
\end{frame}

\begin{frame}[fragile]{\secname}{Definizione}
\begin{overlayarea}{\textwidth}{0.7\textheight}
Per essere una monade, \h/m/ deve soddisfare (oltre alle proprietà funtoriali):
\begin{onlyenv}<+(1)>
\begin{itemize}
\item per ogni tipo \h/a/,
\begin{haskellcode}
join . join == join . map join
    :: m (m (m a)) -> m a
\end{haskellcode}
\begin{diagram}
\h/m (m (m a))/\rar{\h/map join/}\dar["\h/join/" above]&\h/m (m a)/\dar["\h/join/" above]\\
\h/m (m a)/\rar{\h/join/}& \h/m a/
\end{diagram}
\end{itemize}
\end{onlyenv}
\begin{onlyenv}<+(1)>
\begin{itemize}
\item per ogni tipo \h/a/,
\begin{haskellcode}
join . unit == join . map unit == id
    :: m a -> m a
\end{haskellcode}
\begin{diagram}
\h/m a/\rar{\h/unit/}\ar[dr,"\h/id/"]&\h/m (m a)/\dar["\h/join/" above]&\h/m a/\lar["\h/map unit/" above]\ar[dl,"\h/id/" above]\\
&\h/m a/
\end{diagram}
\end{itemize}
\end{onlyenv}
\end{overlayarea}
\end{frame}

\begin{frame}[fragile]{\secname}{La monade \h/Maybe/}
Il funtore \h/Maybe/ ha una struttura di monade.
\pause

\begin{haskellcode}
instance Monad Maybe where
    unit x = Just x
    join Nothing  = Nothing
    join (Just x) = x
\end{haskellcode}
\pause 
\begin{runhaskell}
>>  join Nothing
    Nothing
>>  join (Just Nothing)
    Nothing
>>  join (Just (Just "Rickard"))
    Just "Rickard"
\end{runhaskell}
\end{frame}

\begin{frame}[fragile]{\secname}{La categoria di Kleisli}
Funzioni di tipo \h/a -> m b/ si possono comporre mediante l'operatore di Kleisli.
\pause

\begin{haskellcode}
(<=<) :: Monad m =>
    (b -> m c) -> (a -> m b) -> (a -> m c)
h <=< g = join . map h . g
\end{haskellcode}
\pause

L'implementazione di \h/maternalGF/ è ora immediata.

\begin{haskellcode}
maternalGF :: String -> Maybe String
maternalGF = father <=< mother
\end{haskellcode}
\pause

E anche:

\begin{haskellcode}
maternalGGGF :: String -> Maybe String
maternalGGGF =
    father <=< father <=< father <=< mother
\end{haskellcode}
\end{frame}

\begin{frame}[fragile]{\secname}{La monade \h/List/}
Spostiamoci ora nel contesto delle \emph{multi-valued functions}, i.e. funzioni \h/:: a -> List b/.
\begin{itemize}
\item<2-> Supponiamo di avere una funzione
\haskell/children :: String -> List String/

\begin{onslide}<3->
\begin{runhaskell}
>>  children "Steffon"
    ["Robert", "Stannis", "Renly"]
>>  children "Renly"
    []
\end{runhaskell}
\end{onslide}
\item<4> Vogliamo scrivere
\haskell/grandchildren :: String -> List String/
\end{itemize}
\end{frame}

\begin{frame}[fragile]{\secname}{La monade \h/List/}
Il funtore \h/List/ ha una struttura di monade.
\pause

\begin{haskellcode}
instance Monad List where
    unit x = [x]
    join Nil = Nil
    join (Cons l ls) = l ++ (join ls)
\end{haskellcode}
\pause

\begin{runhaskell}
>>  unit "Renly"
    ["Renly"]
>>  join [[1, 2], [], [2, 3]]
    [1, 2, 2, 3]
\end{runhaskell}
\end{frame}

\begin{frame}[fragile]{\secname}{La monade \h/List/}
Avendo dato una struttura di monade al funtore \h/List/, è automaticamente definita la composizione di Kleisli \h/(<=<)/.
\pause

\begin{runhaskell}
>>  let g = \x -> [x, x + 1, x + 2]
>>  let h = \x -> [x, -x]
>>  g 1
    [1, 2, 3]
>>  (map h . g) 1
    [[1, -1], [2, -2], [3, -3]]
>>  (h <=< g) 1
    [1, -1, 2, -2, 3, -3]
\end{runhaskell}
\pause

Allora possiamo semplicemente definire

\begin{haskellcode}
grandchildren :: String -> List String
grandchildren = children <=< children
\end{haskellcode}
\end{frame}

\begin{frame}[fragile]{\secname}{Conclusione}
Le monadi consentono di comporre funzioni ``con contesto'':
\begin{itemize}[<+(1)->]
\item possibilità di fallimento (\h/Maybe/ o \h/Either String/);
\item non-determinismo (\h/List/);
\item operazioni con stato (\h/State s/);
\item parsing (\h/Parser/);
\item logging (\h/Writer/);
\item interazioni con il mondo esterno (\h/IO/).
\end{itemize}
\pause
\begin{block}{Morale}
In questi (e molti altri) contesti, l'ambiente giusto in cui lavorare sono le categorie di Kleisli.
\end{block}
\end{frame}